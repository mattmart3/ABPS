\documentclass[a4paper,10pt]{article}
\usepackage[utf8]{inputenc}
\usepackage[italian]{babel}
\usepackage{hyperref}
\usepackage{graphicx}
\usepackage{wrapfig}
\usepackage{listings}
\usepackage{float}
\usepackage{bytefield}
\usepackage{fullpage}
\usepackage{algorithm}
\usepackage{algpseudocode}

%opening
\title{ABPS: TED fragmentation fixing}
\author{Matteo Martelli}

\begin{document}

\maketitle


\section{Nelle puntate precedenti}
In questo documento si fa riferimento alla modifiche apportate al componente software TED (Transmis-
sion Error Detector) riguardanti la frammentazione. Per maggiori informazioni su TED e sull’architettura
Always Best Packet Switching (ABPS) si rimanda alla documentazione
precedente. %todo ref

Nello specifico TED offre, tramite una struttura cross-layer, un meccanismo in grado di fornire infor-
mazioni alle applicazioni riguardo l’avvenuta (o mancata) consegna al primo access point dei datagram
UDP trasmessi.

Quando l’applicazione invia un datagram UDP, TED lo marca con un identificativo user-space acces-
sibile dall’applicazione. Successivamente ogni frammento datalink spedito dall’interfaccia di rete viene
associato a quell’identificativo e ad una struttura dati contenenti informazioni sullo stato del frammento.
Quest’ultime vengono poi integrate con le informazioni riguardanti l’avvenuta o mancata consegna (ACK,
NACK) del frammento all’access point e il relativo numero di tentativi di consegna.

Nelle versioni precedenti a questa mancava il corretto supporto per la gestione della frammentazione
in IPv6. Vedremo nella prossima sezione quali modifiche sono state apportate al kernel per il corretto
funzionamento con la frammentazione in IPv6.

Infine nella sezione \ref{sec:app} vedremo le modifiche apportate all’applicazione per la gestione delle notifiche
di TED dei datagram frammentati oltre a qualche esempio d’utilizzo.


\section{Modifiche Kernel}



\section{Modifiche Applicazione}
\label{sec:app}
\subsection{Gestione Asincrona delle Notifiche}
%select vs. epoll

\subsection{Gestione della Frammentazione}
%hashtable, structs, sorting, ricomposizione frammenti

\subsection{Refactoring e Organizzazione}
%json log buttato, both IP versions, struttura file

\subsection{Utilizzo}
%usage, esempio invocazione e output

\end{document}
